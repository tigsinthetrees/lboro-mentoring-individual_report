%! Author = tigsedmonds
%! Date = 04/05/2023
%! Objective = Link their achievement in the presentation to the stuff you said on technical approach

% To what do you attribute success/failure? (10 pts)

\section{Analysis of Achievement}\label{sec:achievement}
    The Belbin analysis showed that the team did not have strong people skills, overall.
    This may have cost them during their viva presentation, as several of the criticisms offered by the assessors regarded their presentation skills.
    In particular, the assessor said that they had too much text on their slides, and that some team members had been reading off the slide during the presentation.
    Their presentation skills had noticeably improved over the in-company meetings, where they had to present their findings on a weekly basis.
    Unfortunately they made mistakes in the viva that they had not made during the company meetings, possibly due to nerves.
    For example, usually they did not use a PowerPoint in company meetings, rather using the Visualiser to project an image of their robot onto the screen, so they could then point to the relevant part of the robot as they spoke.
    It is unclear if the use of a PowerPoint in the final viva was mandated, but if it was they should have practiced with this throughout the project, or at least significantly ahead of the viva.

    As an additional lack of a clearly people-oriented team member meant that none of the team took on an explicit leadership role.
    Observationally, Student B seemed to pick up a lot of these tasks, but by not assigning them to him formally, the associated work may not have been recognised, and little conscious effort was made in this vein.
    This was reflected in their lack of use of formal management tools, and a weakly used project plan.
    However, despite this, the team produced a robot that functioned well (especially in the olympiad), and had several non-essential features to improve the design.
    This is likely because the plan, though not maintained, showed an initial understanding of the level of testing required.
    They did not use the language `interfacing' but it was clear that the risks they identified for the testing stage regarded the interfacing of the subsystems, which was insightful.
    As a result they allocated plenty of time for this, which allowed for time to debug the code they had written, and the robot they had built.

    This debugging process could have been accelerated if they had developed a full set of non-functional requirements, as explored in \autoref{subsec:technical-knowledge}.
    Their verification efforts for their system were therefore somewhat informal, and their testing was very unstructured.
    This let them down when debugging, and was picked up by the examiners during questioning.
    Additionally, it appeared to affect the robot's performance until the end of the project.

    The other issue that the team faced at the end of the project was identifying the correct code to upload to the robot.
    This issue was probably helped by their adoption, and consistent usage, of GitHub, which meant that they had all previous versions of the code to hand, with commit messages to indicate the contents of each version.
    Equally, the use of GitHub may have somewhat complicated the issue by essentially storing many more versions of their code than they could have kept otherwise.
    Regardless, the help that GitHub provided during development is inarguable, and they managed to find a version of their software that worked for the Olympiad.

    Another example of their risk management was revealed during questioning during the viva.
    As \autoref{subsec:management-plan} identified, their allocation of tasks led to major risks if one team member had become ill and been unable to contribute.
    The team showed that they had contingencies to allow another team member to substitute for someone else's absence, by having regular commits to shared file systems, and team meetings to collaborate and update each other on progress.
    In reality, this team was lucky enough to avoid such an eventuality, but their clear risk management strategies fostered behaviours that were generally conducive to performing well.
    The regular meetings meant that all team members were included in the work, and they could resolve issues as a team, and the use of file sharing systems meant that work was backed up, and issues with the file systems could be quickly rectified.


