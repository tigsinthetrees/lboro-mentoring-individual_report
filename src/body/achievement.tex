%! Author = tigsedmonds
%! Date = 04/05/2023
%! Objective = Link their achievement in the presentation to the stuff you said on technical approach

% To what do you attribute success/failure? (10 pts)

\section{Analysis of Achievement}\label{sec:achievement}
    The Belbin analysis showed that the team did not have strong people skills.
    This may have cost them during their viva, as several of the criticisms offered by the assessors regarded their presentation skills.
    In particular, the assessor cited had too much text on their slides, and some team members had read from the slides.
    Their presentation skills had noticeably improved over the weekly company meetings, but unfortunately they made some new mistakes in the viva, possibly due to nerves.
    For example, usually they did not use a PowerPoint in company meetings, but if they needed to use one during the viva they should have practiced with this throughout the project.

    By avoiding nominating a manager, no one was clearly responsible for administration activities.
    This was reflected in their lack of use of formal management tools, and a weakly used project plan.
    However, despite this, the team produced a robot that functioned well (especially in the olympiad), and had several non-essential features to improve the design.
    This is likely because in their initial plan they showed an understanding of the level of testing required.
    They did not use the language `interfacing' but it was clear that the risks they identified for the testing stage regarded the interfacing of the subsystems, which was insightful.
    As a result they allocated plenty of time for this, which allowed for time to debug the system.

    This debugging process could have been accelerated by a set of non-functional requirements, as explored in \autoref{subsec:technical-knowledge}.
    Their verification efforts for their system were therefore somewhat informal, and their testing was very unstructured.
    This was picked up by the examiners during questioning, and resulted in issues that plagued the robot's performance.

    The other issue that the team faced at the end of the project was identifying the correct code to upload to the robot.
    The use of GitHub may have somewhat complicated the issue by storing many more versions of their code than they could have kept otherwise.
    Regardless, the help that GitHub provided during development is inarguable, and they managed to find a version of their software that worked for the olympiad.

    As \autoref{subsec:management-plan} identified, their allocation of tasks had associated risk.
    The team showed that they had contingencies to allow another team member to substitute for someone else's absence, by having regular commits to shared file systems, and team meetings to collaborate.
    In reality, this team was lucky enough to avoid such an eventuality, but their clear risk management strategies fostered behaviours that were generally conducive to performing well.
    The regular meetings meant that all team members were included in the work, and they could resolve issues as a team, and the use of file sharing systems meant that work was backed up, and issues with the file systems could be rectified.


