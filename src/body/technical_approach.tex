%! Author = tigsedmonds
%! Date = 04/05/2023
%! Objective = Evaluate how well they did the technical stuff over the project

% Evaluation of technical approach (20 pts) (5points each)
%   Approach and application to the project
%   Technical knowledge/understanding.
%   Requirements specification
%   Level of effort of Part B


\section{Evaluation of Technical Approach}\label{sec:technical}
    \subsection{Project Approach}\label{subsec:technical-project_approach}
        This team took an agile approach to development, which was suitable due to the nature of the assessment requirements, which placed a large emphasis on finishing the project with a working robot.
        Agile development focuses on getting a working product as fast as possible and adding features with future iterations.
        The team resisted nominating an official team leader, which could have caused issues in a larger team, as a facilitator provides structure.

        The team implemented a version control strategy using GitHub to mitigate the risk identified in \autoref{subsec:management-plan}.
        This gave everyone access to the code, so if one group member fell ill, no work would be lost.
        Initially, the team tried to share code using a Teams space but found this unreliable, and some code disappeared.
        Thankfully, this occurred after they had started using GitHub, so they did not lose work, and this demonstrated the value of version control.

    \subsection{Requirements Specification}\label{subsec:technical-requirements}
        The drawback of the Agile approach was that the team did not spend much time on formal design.
        Requirements were written retrospectively rather than to guide and document the design.
        Ordinarily, requirements ensure that all team members are working towards the same goal.
        They were lucky that the Maze Solver was the task with the clearest design specification provided in the brief, so they could initially use this as an informal set of requirements.

        Later, the mentors provided resources on operational, functional and non-functional requirements.
        The team developed an operational requirement and broke this down into five functional requirements.
        Despite prompting, the team did not produce a set of non-functional requirements.
        This held them back in their testing and verification efforts, as they could not show whether they had met their requirements.

    \subsection{Knowledge and Understanding}\label{subsec:technical-knowledge}
        The team understood that having a modular system was advantageous, but did not grasp the true benefits or risks.
        They understood themselves as a company selling a product, so they saw modularity as a product feature, but did not see that it meant that they could add features and efficiently perform repairs.
        They also missed that building a system in distinct subsystems requires a strict definition of interfaces, to allow independent development.
        This team was modifying the interface of their code with the robot on the day of their viva.

        As a result of this lack of systems thinking, they could not use the features of modularity that would have made it ideal for Agile management: modular subsystems would have allowed iterative development on each subsystem independently.
        For this project, this was not an issue, as there were few subsystems, and the interfaces were simple.
        However, this approach would not work in a system that had to interact with external systems.
        They would have benefited from developing some structured diagrams of their robot.
        In particular, Internal Block Diagrams could have documented the interfaces they were designing, encouraging them to consider the emergent properties of subsystem changes.
        Unfortunately, this project will not have taught them this skill, nor the need for it, as it was not essential for such a simple system.

        Ideally, non-functional requirements specify how the system must work and how well it must perform.
        As the team did not have these standards, testing was haphazard, and they lacked a suite of tests for each subsystem.
        There was no clear understanding of which parts of the system were performing unacceptably, extending fault-finding.
        For example, the team struggled to get the robot to remain equidistant from the maze walls.
        They could not say whether this issue was caused by the requirement, the sensors, or their code.
        They could not find the root cause of this issue, and attempted to resolve it by compensating manually in the code.
        They were partially successful, and their robot navigated mazes accurately, but was not reliably.

    \subsection{Team Effort}\label{subsec:technical-effort}
        The structure of the company meetings meant that the team needed to present their progress weekly.
        This improved their presentation skills and encouraged constant progress.
        As mentioned, this team focused on developing a working robot as fast as possible and added additional features later.
        This meant that they were the first group in their company to be able to show a demonstration of their robot working, and they continued to show demonstrations frequently.

        The team also stretched themselves by implementing a gyroscope at the suggestion of a mentor.
        None of the other maze-solver teams had included anything other than ultrasonic sensors, which were included in the provided robot.
        The gyroscope added complexity to their system, and allowed for more precise turns and testing results showed the benefit of this decision.

        Additionally, they challenged themselves to make the robot as visually appealing as possible.
        Their robot was the tidiest of the maze-solver robots and was among the neatest robots in any team.
        They used JST connectors to avoid loose wires and 3D printed parts to ensure the robot fitted together neatly;
        both required them to learn additional skills.
        Overall, their willingness to learn new skills to add optional features to their project was impressive and served them well in the final product.

