%! Author = tigsedmonds
%! Date = 04/05/2023
%! Objective = % todo: clarify objective for conclusion.tex


\section{Conclusion}\label{sec:conclusion}
    In conclusion, the maze solver team performed well throughout this project.
    In addition to technical skills, they developed a better understanding of systems engineering, which will serve them well in future projects.
    Their consistent effort resulted in a high level of achievement in the final assessments, and overcame some of the issues that were caused by the lack of systems techniques applied.

    The students performed in a manner that was remarkably well predicted by the Belbin tests.
    Student 1 was assigned all tasks relating to software development, and these tasks gave him the freedom to be creative with his solution, without the constraints of a budget.
    Student 2 did end up primarily responsible for the assessment tasks, and appeared to do most of the datasheet and final presentation.
    Student 3 was left at a loose end at some points, but with additional direction he contributed meaningfully to the electrical subsystems, and stretched himself to produce a PCB for the final robot.

    Towards the middle of the project, the mentors found the challenger approach to be especially beneficial.
    It allowed the mentors to suggest areas for improvement, or areas where additional functionality could be added.
    Towards the end of the project, a softer coaching approach was required, as the team was now confident enough to ask direct questions and advice of the mentors.
    The Part Bs managed the stress of the project well, and seemed to find the contributions of and encouragement from the Part Ds both helpful and comforting.