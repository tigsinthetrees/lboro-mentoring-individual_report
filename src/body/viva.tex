%! Author = tigsedmonds
%! Date = 04/05/2023
%! Objective = Analyse performance of the team in the viva and olympiad

% Evaluation of viva presentation (10 pts)
%   Based on demonstration & presentation how they performed?


\section{Evaluation of Viva Performance}\label{sec:viva}
    \subsection{Presentation}\label{subsec:viva-presentation}
        The team only practiced their presentation on the morning of their viva.
        Given that their viva was at 9:00am, this left little time for adjustments.
        They did not take the mentors' offers to practice ahead of time, which limited the usefulness of the comments that could be made.

        Despite this, the team presented their information confidently, and relatively seamlessly.
        Student C seemed the most nervous about the presentation, and answered some questions poorly.
        In particular, he often missed the opportunity to advertise the best parts of their system.
        Luckily, in these situations, Student A stepped in, providing additional information.
        Overall, Student A answered most of the questions, and seemed to have the best understanding of the system, which was in line with observations over the course of the project.
        Student B seemed reluctant to insert himself into questions unless he was specifically asked, which was unfortunate as conversation with the mentors revealed plenty of knowledge.

        Overall, the presentation was comprehensive and professional.
        The addition of branding for their robot made the team feel cohesive, and married well with how much effort they had put into the visual appeal of their robot.
        The presentation gave a good sense of their progress over their project, and highlighted the key features of the work they had done.
        They answered the questions they were asked well, and confidently identified areas for improvement.

    \subsection{Demonstration}\label{subsec:viva-demonstration}
        The team had some issues with their robot in advance of the viva.
        During the run-through, their robot worked on the maze that they had built for themselves, but could not navigate the examiners' maze.
        The team believed this was due to uploading the wrong code.
        They rectified this, but did not have time to re-test ahead of the viva.

        In the viva itself, the robot had issues with both mazes.
        Ultimately it did succeed in navigating the maze provided by the examiners, but it took several attempts.
        The assessor suggested that the inconsistency of the errors suggested that the root cause was noise.
        This could have been resolved through more thorough verification, as identified in \autoref{subsec:technical-knowledge}.

    \subsection{Olympiad}\label{subsec:viva-olympiad}
        Between the viva and the olympiad, the team uploaded yet another version of their code to the robot, which they hoped that would resolve the issues they had faced during the viva.
        In the olympiad, the robot performed well, completing the course twice out of a possible three times, which demonstrated the robot's capabilities far better than during the viva.
        They achieved the fastest single run of the maze, but were penalised for contact with the walls.
        Overall, the team came second out of the five teams, which is admirable, and the team seemed happy with this result.