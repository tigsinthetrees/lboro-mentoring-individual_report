%! Author = tigsedmonds
%! Date = 04/05/2023
%! Objective = Analyse performance of the team in the viva and olympiad

% Evaluation of viva presentation (10 pts)
%   Based on demonstration & presentation how they performed?


\section{Evaluation of Viva Performance}\label{sec:viva}
    \subsection{Presentation}\label{subsec:viva-presentation}
        The team did their first run-through of their presentation on the morning of their viva.
        Given that their viva was at 9:00am, this left little time for improvements, or adjustments, and the team members seemed unclear on who was presenting which part of the presentation.
        They did not take the mentor's offer to practice ahead of time, which limited the usefulness of the comments that could be made.
        It appears that this is because the presentation was made at the last minute.

        However, despite this, the team presented their information confidently, and relatively seamlessly.
        Student C seemed the most nervous about the presentation, spoke very fast, and sometimes missed the key point when answering questions.
        In particular, he often missed the opportunity to advertise the best parts of their system.
        Luckily, in these situations, Student A stepped in, providing additional context after Student C had finished speaking.
        Overall, Student A answered most of the questions posed to the team, and seemed to have the best understanding of the system.
        This was in line with the observations of the mentors over the course of the project.
        Student B was a proficient presenter, if a little quiet.
        He seemed reluctant to insert himself into questions unless he was specifically asked, which was a shame as direct conversation with the mentors revealed that he had a very good base of knowledge, and was able to present it articulately.

        Overall, the presentation was comprehensive and professional.
        The addition of branding for their robot made the team feel cohesive, and married well with how much effort they had put into the visual appeal of their robot.
        The presentation gave a good sense of their progress over their project, and highlighted well the key features of the work they had done.
        They answered the questions they were asked well, openly highlighting areas for future improvement.

    \subsection{Demonstration}\label{subsec:viva-demonstration}
        The team had some issues with their robot in advance of the project.
        In the week before, the robot had accidentally fallen off a table, breaking an ultrasound sensor.
        They were able to get this fixed, but still had some issues with the robot on the morning of their presentation.

        Their robot worked during the run-through on the maze that they had built for themselves, which appears to be the maze they used for testing throughout the project.
        However, the robot could not navigate the maze provided by the examiners during the run-through.
        The team believed this was due to uploading the wrong code to the robot.
        They rectified this, but did not have time to test it ahead of the assessed presentation.

        In the viva itself, the robot had issues with both their own maze and the test maze provided by the examiners.
        The robot would work accurately sometimes, but not reliably.
        Ultimately it did succeed in navigating the maze provided by the examiners, but it required manual selection of the most appropriate algorithm for the maze shown.
        The assessor suggested that the inconsistency of the errors they were facing likely meant the issue was caused by noise, and the issue was ultimately due to the issue with staying in the centre of the path explored in \autoref{subsec:technical-knowledge}.

    \subsection{Olympiad}\label{subsec:viva-olympiad}
        Between the viva and the olympiad, the team uploaded yet another version of their code to the robot.
        They hoped that this would resolve the issues they had faced during the viva, and tested the robot again during this period.

        In the olympiad, the robot performed well, completing the course twice out of a possible three times, which demonstrated the robot's capabilities far better than during the viva.
        They achieved the fastest single run of the maze, but were penalised for contact with the walls.
        Overall, the team came second out of the five teams, which is admirable, and the team seemed happy with this result.
        Two of the other teams were eliminated entirely for failing to complete the course, which highlighted the technical challenge they had undertaken.