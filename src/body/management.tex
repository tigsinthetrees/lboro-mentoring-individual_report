%! Author = tigsedmonds
%! Date = 04/05/2023
%! Objective = % todo: clarify objective for management_evaluation.tex

% Management evaluation (10 pts each)
%   Use of SMART and clear objectives
%   Mapping and Managing the plan


\section{Management Evaluation}\label{sec:management}
    \subsection{Use of Objectives}\label{subsec:management-objectives}
        This team did not formalise any objectives for personal growth, in a SMART or any other form.
        This will make it difficult for them to measure their own performance over this project, and may have limited the amount they learnt over this process.
        Specifically, it is expected that a lack of objectives will make them less aware of which new skills they have learnt and applied during the year, and this may mean they retain fewer of the skills they have improved.

        For some students, informal objectives were observed.
        Student A expressed at the beginning of the project that he had limited knowledge of C++, but was keen to learn.
        As a result, he took the lead on the coding for the project, and has doubtless greatly improved his competence since then.
        \begin{table}
            \centering
            \caption{Project Objective}
            \label{tab:project_objective}
            \begin{tabular}{ccl}
                S & - & To write code to achieve the system aim         \\
                M & - & as measured by the performance in the viva      \\
                A & - & by writing and testing code throughout the year \\
                R & - & in order to meet assessment criteria            \\
                T & - & by the end of the project.
            \end{tabular}
        \end{table} % project objective

        \autoref{tab:project_objective} shows a very high level, formalised objective for the entire project.
        In retrospect, Student A saw that this was the project objective, and by meeting this, achieved his personal objective.
        However, by failing to formalise his goal, he lost the chance to make his personal objective specific to his growth.
        For example, he might have wished to improve his knowledge of coding by learning to code in an object-oriented manner - a useful skill, which would also be applicable to many future modules.
        With this said, his personal objective was at least closely coupled with the project objective, so his achievement can be measured by the feedback from the project.

        Student B also expressed a weakness in coding at the outset of the project.
        However, he did not appear to wish to strengthen this skill, and did not contribute overly much to the software of the final system.
        From observation, he contributed significantly to the project management tasks, and to the deliverables, such as the datasheet.
        This is in line with the Belbin test's indications of his strengths.
        Thus, he may have stayed in his comfort zone during this project, and may have learnt less as a result.
        In person interactions revealed Student B to be relatively shy, and lacking in confidence.
        It is possible that the pressure of teammates relying on his contributions made him reluctant to try new tasks.
        In his project next year, he will have to write and reflect on formal objectives for assessment purposes.
        He will likely find this process very helpful, and having concrete requirements for his development may help him to take some risks in the tasks he undertakes.

        Finally, Student C took the lead in the development of the electrical aspects of the project, and thus he may have wished to develop his abilities in this area.
        In particular, late in the project, he seemed to wish to learn how to design and print PCBs.
        However, this is an area of the project constrained by a budget.
        Therefore, endless experimentation was not possible, and a clear personal objective would have been especially helpful to direct his growth.
        Understanding this objective at the beginning of the project could have led to allocation of part of the budget for this task.
        In this instance, the budget was not so restrictive as to prevent him from his objective, but it is easy to conceive of a situation where this would not have been possible.

    \subsection{Mapping and Managing the Plan}\label{subsec:managemement-plan}
        The team was required to create a Gantt chart showing their project plan for their initial presentation.
        This chart was created and then never updated.
        This was anticipated by the mentors, as Gantt charts are difficult to maintain, as in order to be used effectively they must be updated very regularly.
        Additionally, sub-steps for the planned tasks are difficult to show, and are thus typically excluded.
        Small unanticipated changes or delays are difficult to include, especially if, as in the case of this group, the Gantt chart is constructed by colouring cells in Excel.
        The result was a visually appealing diagram that did not function to manage their project.

        This team employed an Agile approach to development.
        Their approach likely both helped and hindered their management efforts.
        Developing the system in this way reduced the likelihood of maintaining the Gantt chart effectively.
        This is because organising each sprint allocates many subtasks, which are difficult to show and maintain in a Gantt chart.
        Additionally, the fast pace of this method makes time-intensive project management tasks, like maintaining a Gantt chart, unappealing.
        Agile development allowed them to achieve a working solution very quickly, and subsequent sprints could then be used to add functionality.
        This reduced the impact of otherwise lacking project management, because they reviewed the next steps for development after each sprint.
        However, this technique only worked because of the relative simplicity of the project.
        True Agile development requires careful management of the order that functionalities need to be added.
        However, in this case, there were very few functionalities that the system needed to achieve, so the order of development was clear.

        Initially, the team resisted clear allocation of tasks to different team members, attempting instead to work highly collaboratively.
        This is a result of their decision to avoid a hierarchical team structure, as they had no manager to oversee that tasks were clearly delineated and make responsibility clear.
        Ultimately, the team found that they were getting in each other's way, and subsequently reorganised themselves.
        They allocated all software tasks to Student A, all mechanical tasks to Student B, and all electrical tasks to Student C\@.
        This decision was considered inadvisable, as it led to an uneven distribution of work, as this task was primarily software based.
        Additionally, it added risk, as if one team member became indisposed, their domain could not progress easily.
