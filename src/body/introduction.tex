%! Author = tigsedmonds
%! Date = 04/05/2023
%! Objective = ???

\section{Introduction}\label{sec:introduction}
    During Part B, students in Electrical and Electronic, Robotics and Mechatronic Control and Electrical and Computer Systems Engineering have to participate in a group project to develop a robot.
    The cohort is split into five companies, and each company is split into groups of three to four students.
    Each group will attempt one of the four challenges: maze solver, orienteering, gymnastics or pied piper.
    At the end of the year, the teams produce a datasheet for their robot, present and defend their work in a viva, and compete with other teams in the olympiad.

    \subsection{Team Overview}\label{subsec:intro-team}
    This report primarily follows the mentoring and progress of a maze solver team.
    The robot needed to be able to navigate a maze independently, finding its way through as fast as possible.
    In the olympiad, the robot would run through the maze three times, and penalties would be applied for each time the robot failed to find a way out, or touched a wall.

    The team was initially analysed using a Belbin test, and comparing the team roles. % todo: include belbin in appendicies
    % todo: reference first report
    The Belbin tests showed that Student 1 scored most highly on the `plant' role, Student 2 was indicated to be a `shaper', and Student 3 was shown to be a jack-of-all-trades, with a slight preference for the `coordinator' role.
    Mentoring to this team was provided by two Part D mentors, who adopted a coach and challenger mentoring style, in order to push the team in directions they may not have otherwise considered, and encourage them to stretch themselves.
%
%Oscar = Student A
%Paul = Student B
%Uzair = Student C
%
%project is primarily code based
%
%overview of belbin