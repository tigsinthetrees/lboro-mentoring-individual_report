%! Author = tigsedmonds
%! Date = 04/05/2023
%! Objective = ???

\section{Introduction}\label{sec:introduction}
    During Part B, students in Electrical and Electronic, Robotics and Mechatronic Control and Electrical and Computer Systems Engineering have to participate in a group project to develop a robot.
    The cohort is split into five companies, and each company is made up of smaller groups of three to four students.
    Each group will attempt one of the four challenges: maze solver, orienteering, gymnastics and pied piper.
    They will spend the year developing on a base robot to complete their task optimally.
    At the end of the year, the teams produce a datasheet for their robot, present and defend their work in a viva, and compete with other teams in the olympiad.


    \subsection{Team Overview}\label{subsec:intro-team}
    This report primarily follows the mentoring and progress of a maze solver team.
    The robot needed to be able to navigate a maze independently, finding its way through as fast as possible.
    In the olympiad, the robot would run through the maze three times, and penalties would be applied for each time the robot failed to find a way out, or touched a wall.

    The team was initially analysed using a Belbin test, and comparing the team roles. % todo: include belbin in appendicies
    % todo: reference first report
    The Belbin tests showed that Student 1 scored most highly on the `plant' role, and he was therefore expected to provide most of the creative impetus for the group, but risked lacking focus on the task at hand.
    Student 2 was indicated to be a `shaper', and therefore was expected to be the most organised and focused, but also the most prone to stress.
    The evaluation of Student 3 showed him to be a jack-of-all-trades, with a slight preference for the `coordinator' role.
    A risk of Student 3 under-contributing to the project was identified, and some mitigations were suggested.

    Mentoring to this team was provided by two Part D mentors, who aimed to provide advice on project design amd management, point the team in the direction of any technical information they might need, and evaluate their progress.
    Initial meetings with the team showed them to be enthusiastic and motivated, and they seemed to have a strong grasp of what was required of them technically.
    Moreover, they seemed keen to take any advice offered by the mentors, and were keen to learn the skills they did not already have.
    Therefore, the mentors adopted a coach and challenger mentoring style, in order to push the team in directions they may not have otherwise considered, and encourage them to stretch themselves.
%
%Oscar = Student A
%Paul = Student B
%Uzair = Student C
%
%project is primarily code based
%
%overview of belbin