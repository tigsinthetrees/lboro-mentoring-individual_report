%! Author = tigsedmonds
%! Date = 04/05/2023
%! Objective = todo: clarify objective for reflection.tex

% Reflection (20 points)
%   What did you bring to the Part B group? (5 points)
%   Was this useful to the Part B group? (5points)
%   What did you learn/gain as a mentor(self reflection)? (10 points)

\section{Reflection}\label{sec:reflection}
    \subsection{Contribution to Part Bs}\label{subsec:reflection-partBs}
        I provided the team with the resources on requirements engineering and offered feedback on their efforts.
        I believe they found this useful in the viva, as it gave a clear structure on which to base the architecture specification.
        By the end, the team also understood why non-functional requirements would have been helpful, and  questions from the examiner cemented this.
        While this did not help this year, I hope these skills will be useful to them in future work.

        All the students in this team will be taking a module using the Unified Modelling Language (UML) to document object-oriented programs next year.
        On reflection, I should have framed the information I provided to them against the content of that module.
        I suggested systems engineering techniques and using the Systems Modelling Language (SysML) as I believe it is the most applicable to this task.
        However, the team may have found it more helpful for me to show them how to use UML diagrams to achieve a similar effect.
        SysML is based on UML, but the differences may have been unnecessarily confusing and reduced the likelihood of adoption for this project.

        Much of my contribution to the project was through the form of questions during presentations in the company meetings.
        In particular, my questions focused on testing the capabilities of the robot and on project planning.
        As previously discussed, my team's project planning could have been improved, so this Socratic feedback method was not as helpful as I had hoped.
        We suggested a project management tool, Jira, which is ideal for Agile project management but is quite complex and has a steep learning curve.
        On reflection, the team should have used GitHub projects, as this would have built project management into a tool they already used.
        GitHub projects allow the construction of Kanban boards, and Milestones can be used to organise tasks within a project repository.
        These features would have added little overhead to the time spent on management but may have focused their efforts.

    \subsection{Personal Growth}\label{subsec:reflection-personal}
        Working with the Part Bs over this year has been very satisfying: I have enjoyed seeing their progress and helping them as I have been able.
        It was gratifying to see the group implement and benefit from my suggestions and to see them succeed in the olympiad.

        I have learnt a lot about conveying the knowledge I have effectively.
        For example, I struggled to get the Part Bs to complete their Belbin tests early in the project.
        I tried to communicate the steps over text, but the students still needed clarification.
        I then showed them in person and sat with them while they completed the test, which worked much better because I could demonstrate the steps.
        This knowledge was also helpful in working with my Part D teammates, when I had a similar issue ahead of the first group report.
        As a result of my experience with the Part Bs, I explained the process in person in the first instance which worked well.

        Despite the initial failure, I have improved my skill in written communication.
        I had written the instructions for the Belbin tests in a Discord message. % todo: add figure
        This may have been difficult to understand and apply, as I did not show the steps in a numbered list.
        Therefore, when it came time to explain how to write requirements, I approached it differently.
        I used a set of slides on requirements engineering which clearly illustrated the steps, and provided an example of requirements I had written for an entirely different system.
        I was told by the Part Bs that the included example requirement specification was particularly helpful, as it illustrated the correct language.

        At the outset of the module, I struggled with providing applicable evaluations to the teams during the company meetings.
        I found the information presented overwhelming and struggled to see what was missing.
        The most helpful technique I developed to manage this was noting the broad topics I expected them to cover before each presentation.
        Typically, my headings were `progress against the plan', `tangible progress', `things that went wrong' and `next steps'.
        I could then add notes as they spoke, even if they switched back and forth between topics. % todo: sample
        This made it clear which areas had been missed, and I could usually find a question to ask.

        Providing positive public feedback was more effective than suggesting corrections.
        For example, one week, two teams presented code samples in their PowerPoints.
        This is a poor way to communicate programming, and the text was too small to read, but I planned to mention this to the company at the end of the meeting.
        When the next team presented and showed how their code worked with a flow chart, I instead congratulated them publicly for this and explained why it was more effective communication.
        The following week, there were no code samples in the teams' presentations, and there were a few rudimentary flow charts.

        Over my degree, I have worked on increasingly complex systems, and it has been easy to lose sight of how the techniques I apply benefit the project.
        This showed me how much I have learnt during the two years since Part B and re-highlighted how to get the most benefit out of the techniques.
        Ultimately, it has made me wish to work on a similar project to reinforce the fundamental applications of systems engineering.
