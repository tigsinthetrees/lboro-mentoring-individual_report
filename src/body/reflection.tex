%! Author = tigsedmonds
%! Date = 04/05/2023
%! Objective = todo: clarify objective for reflection.tex

% Reflection (20 points)
%   What did you bring to the Part B group? (5 points)
%   Was this useful to the Part B group? (5points)
%   What did you learn/gain as a mentor(self reflection)? (10 points)

\section{Reflection}\label{sec:reflection}
    \subsection{Contribution to Part Bs}\label{subsec:reflection-partBs}
        As shown throughout \autoref{subsec:technical-requirements}, the team lacked a working knowledge of requirements engineering.
        I provided them with the resources on requirements engineering that had been provided to me in second year, and offered feedback on their efforts.
        They took this offer up quite late in the project, but they did develop a set of operational and functional requirements.
        I believe they found this useful in the presentation of their project in the viva, as it gave them a clear structure on which to base the architecture specification.
        By this point in the project, the team also seemed to understand why non-functional requirements would have been helpful, and certainly the questions from the examiner cemented this.
        While they could not apply that in this project, I hope that these skills will be useful to them in future work.

        I did not teach them all the systems engineering techniques that they could have used, as this was beyond my role as a mentor, but analysis of the group early on showed good levels of initiative.
        Therefore, I hope that the resources I provided them with on system modeling and interface design will be useful in future years.
        All the students in this team will be taking a module which uses the Unified Modelling Language (UML) to document object-oriented programs next year, which will place the information I have provided in better context.
        On reflection, I should have framed the information I provided to them against the content of that module.
        I suggested systems engineering techniques, and using the Systems Modelling Language (SysML) as I believe it is the most applicable to this task, and contains some features that would have been especially helpful to them.
        However, the team may have found it more helpful for me to show them how to use UML diagrams to achieve a similar effect.
        SysML is based on UML, but the differences may have been unnecessarily confusing, and reduced the likelihood of them using the techniques in this project.

        In addition to advice on Systems Engineering, I advised them on technical writing.
        This module has only one written submission, the technical datasheet, but I believe that I provided useful information on appropriate language to use, and on formatting the document.
        The Part Bs found this useful, and implemented many of the suggestions I made.

        A lot of my contribution through the project was through the form of questions during presentations in the company meetings.
        In this way, I contributed to all the teams in the company.
        In particular, my questions focused on testing the capabilities of the robot, and on project planning.
        As previously discussed, I believe that my team's project planning could have been improved, so perhaps this socratic method of feedback was not as helpful as I had hoped.
        We did suggest a project management tool that the teams could have used to manage their progress, but perhaps I should have focused on small interventions.
        The tool we proposed is Jira, which is ideal for Agile project management, but is quite complex, and has a steep learning curve.

        On reflection, the team should have used GitHub projects, as this would have built project management into a tool they were already using.
        GitHub projects allow the construction of Kanban boards, and Milestones can be used to organise tasks within a project repository.
        These features would have added little overhead to the time spent on management, but may have focused their efforts.

        Additionally, I should have spent more time at the beginning of the project working with my Part Bs on developing suitable objectives.
        This may have helped them to gain more from the process of this project, and made clearer the personal progress they had made.
        In recompense, I spoke with each team member to highlight the skills in which I have noticed significant improvement.

        Overall, this team was quite self-sufficient.
        They worked well as a team and did not face much conflict, which meant that they did not need a huge degree of support.
        Additionally, they were very technically competent, and thus did not need my input in this way either.
        This is very fortunate, as I did not do this module when in second year, and would have had to learn alongside them.

    \subsection{Personal Growth}\label{subsec:reflection-personal}
        Working with the Part Bs over this year has been very satisfying.
        I have enjoyed seeing their progress, and helping them as I have been able.
        It was very rewarding to see the group implement and benefit from my suggestions, and to see them succeed in the olympiad.

        Acting as a mentor has helped to develop my interpersonal skills.
        I have learnt a lot about conveying the knowledge I have effectively.
        For example, early in the project I struggled to get the Part Bs to complete their Belbin tests.
        I tried to communicate the steps for this over text, but the students still seemed confused.
        I then showed them how to complete the spreadsheet in person, and sat with them while they did this.
        This method of communication worked much better, which I think is because I could explain it better when I could demonstrate the steps.
        This knowledge was also helpful in working with my Part D teammates.
        I encountered a similar resistance to completing the Belbin tests in my team ahead of the first report.
        As a result of my experience with the Part Bs, I decided to explain the process in person in the first instance, and found this worked well.

        Despite this initial failure in written communication, I feel that I have improved my ability to explain things over writing.
        For the Belbin tests, I had written out the instructions in a message over Discord. % todo: add figure
        This may have been difficult to understand and apply, as I did not show the steps in a clearly numbered list.
        Therefore, when it came time to explain how to write requirements, I approached the task differently.
        I used a set of slides that had been provided to me as a second year student, and an example of requirements I had written for an entirely different system.
        By showing the information in slides, the steps to complete were clearly illustrated.
        I was told by the part Bs that the included example requirement specification was particularly helpful, as it showed them the right kind of language to use.

        At the outset of the module, I struggled with providing useful evaluation to the teams during the company meetings.
        I found the amount of information presented somewhat overwhelming, and struggled to sort through it and see what was missing.
        I have since developed techniques that help me to manage this.
        The most helpful technique was in noting a set of broad topics that I expected them to cover in each presentation.
        Typically, my headings were `progress against the plan', `tangible progress', `things that went wrong' and `next steps'.
        By having these written before each team started presenting, I could add notes as they spoke, even if they switched back and forth between topics. % todo: sample
        It was then clear which areas had been least covered, and I could usually see a clear question to ask.

        I also found that providing public positive feedback was often better than suggesting an improvement.
        For example, one week, the first two teams to present showed samples of their code in their PowerPoints.
        I think this is a poor way to communicate programming, and the text was too small to read at any rate, but I planned to mention this at the end of the meeting as a general point.
        I planned to do it in this way as I was not the primary mentor for either of the teams presenting, and I didn't wish to make them feel attacked.
        When the next team presented, and showed how their code worked with a flow chart, I instead decided to publicly congratulate them for this, and explained why it was a better method of communication.
        The next week, there were no code samples in the presentations for any of the teams, and there were a few rudimentary flow charts.

        Finally, I feel I have gained a greater insight into the benefits of systems engineering.
        During this project, I have watched my team attempt this task with little knowledge of systems techniques, and it was enlightening to see the places where they clearly could have helped.
        Over my degree, I have worked on increasingly complex systems, and it has been easy to lose sight of how the techniques I apply specifically benefit the project.
        This both showed me how much I have learnt during the two years since Part B, and re-highlighted how to get the most benefit out of the techniques.
        Ultimately, it has made me wish to work on a similar project, to reinforce the fundamental applications of systems engineering.
        I intend to do just such a project over my summer break!

